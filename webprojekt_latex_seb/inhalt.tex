\chapter{Einleitung}
\label{Einleitung}
Im Rahmen der Vorlesung \textit{Webprogrammierung} galt es als Projekt, eine Website zu erstellen, welche ein Schulsportfest in freier Art und Weise unterstützt. Dabei mussten verschiedene Technologien aus dem Bereich der Webprogrammierung eingesetzt werden, um für diese eine Grundverständnis zu entwickeln.
\par
Die in dieser Ausarbeitung behandelte Internetseite unterstützt Teilnehmer und Teilhaber vor, während und nach dem Fest: So kann sich z.B. ein Auswärtiger vor oder am Tage des Sportfestes eine Anfahrtsbeschreibung anschauen, Wettkämpfer können danach eigene Bestwerte betrachten. Auch entstandene Fotos sind in einer Gallerie für jedermann zugänglich.
\par
Diese Arbeit beschreibt die benutzten und geforderten Technologien, indem zuerst jene der Benutzeroberfläche erläutert werden. Danach werden die Technologien für die Client-Server-Kommunikation und der Datenhaltung durchleuchtet.
TODO Gliederung

\chapter{Benutzeroberfläche}
\label{Benutzeroberfläche}

\section{HTML zur Erstellung der Website}
\label{HTML zur Erstellung der Website}
Das Grundgerüst der kompletten Benutzeroberfläche bildet eine \textit{"'index.html"'}-Datei. Als One-Pager ist sie die einzige HTML-Datei und beinhaltet somit sämtliche Anzeige- und Steuerelemente des Projektes.
\par
Dazu zählen:
\begin{itemize}
	\item eine Navigationsbar
	\item ein Banner
	\item eine Übersicht der Wettkämpfe und Dialoge zur Anzeige der Wettkampfergebnisse
	\item eine filterbare Gallerie
	\item einer Anfahrtsbeschreibung mit Google Maps
	\item Kontaktinformationen
	\item ein Bereich für den Log In, die Registrierung \item ein kleines Spiel zum Zeitvertreib
	\item das Heise Plugin für Social Buttons
	\item ein Footer mit der Serverzeit	
\end{itemize}
Auf viele dieser Punkte wird im weitern Verlauf der Arbeit eingegangen. Zudem befinden sich im Anhang Bilder von den Bereichen der Seite, welche nicht weiter genauer behandelt werden.

\section{CSS - Bootstrap}
\label{CSS - Bootstrap}
Zur optischen Verschönerung der Seite wurde das CSS Framework Bootstrap von Twitter eingebunden, was durch ein \textit{<script>}-Tag im \textit{<head>} importiert wurde. Angewendet wird das Framework dann durch Benutzen seiner CSS-Klassen auf das gewünschte HTML-Element. Bspw. lassen sich durch die Klasse \textit{"'btn-lg"'} visuell ansprechende, große, farbige Buttons mit sattem Text und abgerundeten Ecken erzeugen (auch zu sehen in Abbildung \vref{fig:responsive}).
\par
Darüber hinaus wurden für dieses Projekt allerdings auch eigene CSS-Klassen geschrieben, auf welche an dieser Stelle aber nicht genau eingangen werden.

\subsection{Kompatibilität mit mobilen Endgeräten}
\label{Kompatibilität mit mobilen Endgeräten}
Ein weiterer Vorteil des Twitter Bootstrap ist, dass die Website \textit{"'responisve"'} ist, was bedeutet, dass sich die Seite der Fenstergröße anpasst. Dadurch ist sie auch für mobile Endgeräte kompatibel: Bei Verkleinerung des Fensters ordnen sich die Seitenelemente automatisch neu an und die Navigationsbar oben wird durch einen Menü-Button komprimiert (siehe Abbildung \vref{fig:responsive}).

\begin{figure}[!h]
	\makebox[\textwidth]{ 
		\includegraphics[scale=0.5]{img/responsive.png}}
	\caption{Responsive Design der Seite}
	\label{fig:responsive}
\end{figure}

\section{Clientseitiges Javascript für benutzerspezifische Seiteninhalte}
\label{Clientseitiges Javascript}
Um Frontend-seitige Anpassungen der Website vorzunehmen, ohne die komplette Seite neu laden zu müssen, wird JavaScript verwendet. In diesem Projekt wird es unter anderem verwendet, um zu überprüfen ob ein Nutzer angemeldet ist, und um daraus resultierend Wettkampfergebnisse des angemeldeten Benutzers anzuzeigen.
\par
Durch klicken auf den Knopf für die Ergebnisse wird dabei nicht nur ein Dialog aufgerufen, sondern auch eine Funktion ausgelöst (siehe Abbildung \vref{fig:javaScript}). Die globalen Variablen, welche in dieser Funktion benutzt wird, ist beim Aufruf der Seite \textit{null}, und wird erst nach erfolgreicher Anmeldung des Nutzers mit einem Wert belegt (siehe Kapitel \vref{Konsumieren der Daten im Frontent mit AJAX-Calls}).

\begin{figure}[!h]
	\makebox[\textwidth]{ 
		\includegraphics[scale=0.8]{img/javaScript.png}}
	\caption{Skript zur Darstellung der Wettkampfsergebnisse}
	\label{fig:javaScript}
\end{figure}

\section{Anfahrtsbeschreibung per Google Maps}
\label{Anfahrtsbeschreibung per Google Maps}
Für den Fall, dass der Nutzer der Website von außerhalb des Ortes der Schule kommt, wurde eine Anfahrtbeschreibung mittels Google Maps eingebunden (siehe Abbildung \vref{fig:googleMaps}).

\begin{figure}[!h]
	\makebox[\textwidth]{ 
		\includegraphics[scale=0.5]{img/googleMaps.jpg}}
	\caption{Anfahrtbeschreibung mit Google Maps}
	\label{fig:googleMaps}
\end{figure}

Auf dem Bild ist nicht nur der Standort der Schule vermerkt, sondern darüberhinaus kann der User über ein Eingabefeld links oben seine Abfahrtsadresse eingeben, sodass die optimale Route auf der Karte angezeigt wird. Des Weiteren wurde mit Hilfe der Autocomplete Funktion der Google Maps API eine automatische Vervollständigung der Adresse bei Eingabe in das Adressfeld implementiert, wie auf dem Screenshot zu sehen ist.
\par
Verwirklicht wurde die Einbindung, indem zuerst die Google Maps API und die Google Maps API für die Google Places per \textit{<script>}-Tag in den \textit{<head>} der Website importiert wurde (siehe Abbildung \vref{fig:googleMapsAPI}).

\begin{figure}[!h]
	\makebox[\textwidth]{ 
		\includegraphics[scale=0.8]{img/googleMapsAPI.png}}
	\caption{API Einbindung von Google Maps}
	\label{fig:googleMapsAPI}
\end{figure}

Danach wird über eine JavaScript-Funktion, ebenfalls im <head>-Bereich die Map erstellt (siehe Abbildung \vref{fig:googleMapsInit}).

\begin{figure}[!h]
	\makebox[\textwidth]{ 
		\includegraphics[scale=1]{img/googleMapsInit.png}}
	\caption{\textit{initMap}-Funktion zur Erstellung der Karte}
	\label{fig:googleMapsInit}
\end{figure}

Im zweiten Teil der selben Funktion wird die Autocomplete-Funktion für das Input-Feld implementiert und der Marker für die eingegebene Adresse auf der Karte gesetzt (siehe Abbildung \vref{fig:googleMapsInit2}).

\begin{figure}[!h]
	\makebox[\textwidth]{ 
		\includegraphics[scale=1]{img/googleMapsInit2.png}}
	\caption{Autocomplete-Funktion und Adressmarker}
	\label{fig:googleMapsInit2}
\end{figure}

Die Route zwischen der Abfahrtsadresse und der Schule wird im letzten Teil der Funktion erstellt und angezeigt. Vor Schließen des \textit{<script>}-Tags wird noch ein Event-Listener hinzugefügt, welcher auf das Aufrufen der Seite wartet und bei diesem dann die Karte zeichnet, indem dann die Funktion \textit{initMap} ausgeführt wird (siehe Abbildung \vref{fig:googleMapsInit3}).

\begin{figure}[!h]
	\makebox[\textwidth]{ 
		\includegraphics[scale=1]{img/googleMapsInit3.png}}
	\caption{Routenerstellung}
	\label{fig:googleMapsInit3}
\end{figure}

Um nun die Karte auch tatsächlich in der Website anzeigen zu können, wird zu guter Letzt noch \textit{div}-Container mit der Id \textit{"'map"'} an der Stelle, wo die Karte hingezeichnet werden soll, erstellt (siehe Abbildung \vref{fig:googleMapsDiv}). Dieses \textit{div} wird dann durch die Funktion \textit{initMap} gefüllt. Dafür wird bei Deklaration der Karte über einen Id-Selektor das entsprechende \textit{div} ausgewählt, in welchem die Google Map erscheinen soll (siehe Abbildung \vref{fig:googleMapsInit} (ganz oben)).

\begin{figure}[!h]
	\makebox[\textwidth]{ 
		\includegraphics[scale=1]{img/googleMapsDiv.png}}
	\caption{Container für die Karte}
	\label{fig:googleMapsDiv}
\end{figure}

\section{Einbindung von Social Buttons mittels des Heise Plugins}
\label{Einbindung von Social Buttons mittels des Heise Plugins}
Um soziale Netzwerke der Schule oder des Sportfestes auf der Seite verlinken zu können, wurden mittels des Heise Plugins Social Buttons eingebunden (Facebook, Twitter und Google+). Der Vorteil bei diesem Plugin liegt darin, dass der Nutzer selbst bestimmen kann, ob er diese Buttons benutzen möchte, oder komplett für die Seite deaktivieren will. Der Sinn dahinter liegt in der Datenschutzproblematik bei der Einbindung der "`Like"'-Buttons von Facebook und Co.: Das Einbinden dieser Buttons erfolgt über einen iFrame, welcher von Facebook und Co. selbst zur Verfügung gestellt wird. Der iFrame enthält Code, der veranlasst, dass die URL der Seite, Cookies der Seite an Facebook geschickt wird. Ist der Anwender zudem gleichzeitig in einem anderen Fenster bei Facebook angemeldet, so schickt das iFrame zusätzlich Sitzungs-Id mit, wodurch Facebook einen Webseitenaufruf einer konkreten Person zuordnen kann.
\par
Damit das also nicht passiert, dürfte die Seite entweder keinerlei Elemente von Facebook, Twitter und Google+ behinhalten, oder das Heise Plugin verhindert eben genau dies, indem all diese Elemente zunächst bei Aufruf der Seite deaktiviert sind. Möchte der Nutzer nun einer der Funktionen der sozialen Netzwerke nutzen, so muss er die entsprechende erst über das Plugin aktivieren.
\par
In die Website integriert wurde das Plugin, indem zuerst das, von Heise zur Verfügung gestellte Skript des Plugins im \textit{<head>} geladen wird und ein HTML-Element durch ein weiteres Skript gefüllt wird (siehe Abbildung \vref{fig:heisePlugin1}).

\begin{figure}[!h]
	\makebox[\textwidth]{ 
		\includegraphics[scale=1]{img/heisePlugin1.png}}
	\caption{Integrieren des Heise Plugins}
	\label{fig:heisePlugin1}
\end{figure}

Durch ein \textit{div}-Element mit Id \textit{"'socialshareprivacy"'} (wie es in Abbildung \vref{fig:heisePlugin1} selektiert wird) vor dem Footer der Seite wird das Plugin dann am Ende der Seite angezeigt (siehe Abbildung \vref{fig:heisePlugin2}).

\begin{figure}[!h]
	\makebox[\textwidth]{ 
		\includegraphics[scale=1]{img/heisePlugin2.png}}
	\caption{Heise Plugin in der Website}
	\label{fig:heisePlugin2}
\end{figure}

\chapter{Client-Server-Kommunikation}
\label{Client-Server-Kommunikation}

\section{Anmeldefunktion per AJAX-Calls}
\label{Anmeldefunktion per AJAX-Calls}
Ist ein Nutzer der Website gleichzeitig auch Teilnehmer des Sportfestes, was so viel bedeutet, dass für ihn Ergebnisse für Weitsprung, Sprint etc. in der Datenbank hinterlegt sind, kann er sich jene Ergebnisse anzeigen lassen, indem er sich anmeldet. Um diese Information von der Datenbank an das Frontend kommunizieren zu können bedarf es an einem Server, der sich mit der Datenbank verbindet und an einer Technologie, welche sich die benötigten Informationen von dem Server holt und an das Frontend bringt. 
\par
Für diese Website wurde ersteres mit einem REST-Service mit node.js und zweiteres mit jQuery AJAX-Calls realisiert, die den REST-Service konsumieren.

\subsection{Bereitstellung eines REST-Service mit node.js}
\label{Bereitstellung eines REST-Service mit node.js}
Die Anbindung der MySQL-Datenbank wurden in node.js mit Hilfe des Plugins \textit{express} und \textit{mysql} verwirklicht. Nach deren Einbindung wird eine Datenbankverbindung aufgebaut und ein REST-Service erstellt, welcher sich über die URL \textit{"'localhost:3000/getUser/[username]/[password]} als GET-Request aufrufen lässt und dann die kompletten Userinfos als JSON-Objekt zurückgibt (siehe Abbildung \vref{fig:nodeServer1}).

\begin{figure}[!h]
	\makebox[\textwidth]{ 
		\includegraphics[scale=1]{img/nodeServer1.png}}
	\caption{Datenbankverbindung und REST-Service im node.js-Server}
	\label{fig:nodeServer1}
\end{figure}

Zusätzlich wurden noch vor der Erstellung der Datenbankanbindung ein besonderer HTTP-Header erstellt, um das Cross-Origin-Problem zu beheben, durch welches manche Browser AJAX-Calls unterbinden, welche auf externe APIs/ Server zugreifen (siehe Abbildung \vref{fig:nodeServer2}).

\begin{figure}[!h]
	\makebox[\textwidth]{ 
		\includegraphics[scale=1]{img/nodeServer2.png}}
	\caption{Erstellen des Headers für Cross-Origin-Aufrufe}
	\label{fig:nodeServer2}
\end{figure}

\subsection{Konsumieren der Daten im Frontent mit AJAX-Calls}
\label{Konsumieren der Daten im Frontent mit AJAX-Calls}

Meldet der Nutzer sich nun über folgendes Log-In-Fenster im unteren Teil der Website an...

\includegraphics[scale=0.5]{img/logIn.png}
TODO
%...so kann er sich seine Wettkampfergebnisse anzeigen lassen (siehe Abbildung \vref{fig:wettkampfErgebnisse}).
\begin{figure}[!h]
	\makebox[\textwidth]{ 
		\includegraphics[scale=0.55]{img/wettkampfErgebnisse.png}}
	\caption{Wettkampfergebnisse-Dialog}
	\label{fig:wettkampfErgebnisse}
\end{figure}

Dies geschieht durch einen AJAX-Call, welcher bei der Anmeldung ausgeführt werden (siehe Abbildung \vref{fig:ajaxCall}).

\begin{figure}[!h]
	\makebox[\textwidth]{ 
		\includegraphics[scale=1]{img/ajaxCall.png}}
	\caption{AJAX-Call}
	\label{fig:ajaxCall}
\end{figure}

Durch den AJAX-Call werden globale JavaScript Variablen gesetzt, welche dann bei Aufruf des Dialogs abgefragt werden. Sind die Variablen bei Dialogaufruf leer bzw. ist der Nutzer nicht eingeloggt, so steht anstelle seines Wettkampfergebnisses ein Hinweis, dass der User sich für diese Information anmelden muss.

\chapter{Fazit}
\label{Fazit}
Nach Erstellung eines umfangreicheren und in sich abgeschlossenen Projekts, wie es bei dem Projekt dieser Ausarbeitung der Fall ist, ist es mehr oder weniger garantiert, ein fundiertes Grundverständnis trotz der hohen Anzahl an Anforderungen für die verwendeten Technologien zu entwickeln. 
\par
Durch die hohe Anzahl und Vielfältigkeit der technologischen Anforderungen wird allerdings auch sehr schnell deutlich: Es gibt Technologien, welche sich relativ unkompliziert umsetzen lassen, und welche, die umfangreicher und zeitaufwändiger sind. 
\par
Folglich reicht für die erstere Art ein Grundverständnis vollkommen aus. Ein Beispiel hierzu bietet das Heise Plugin: Durch ein kleines Skript und einem zugehörigen HTML-Element ist das Plugin schon komplett eingebunden. Die Funktionalitäten des Plugins können mit dem Grundverständnis darüber vollkommen ausgeschöpft werden.
\par
Anders hingegen ist das bei komplexeren Anforderungen, wie z.B. node.js oder Google Maps. Mit einem Grundverständnis lassen sich zwar einigermaßen schnell eine Datenbankverbindung, ein Server oder eine Karte erstellen - jedoch sind node.js und die Google Maps API wesentlich mächtiger und umfangreicher. 
\\
Mit node.js können bspw. \textit{"'Connection-Pools"'} zur Optimierung von gleichzeitigen Datenbankverbindungen verwendet werden und Websockets aufgebaut werden. 
\\
Die API von Google Maps bietet umfassendere Kartenfunktionen, wie Streetview oder Satellitenkarten.
\\
Bei solchen Anforderungen ist ein tiefgründigeres Wissen über die Technologien von Nöten, um sie wirklich zu meistern.
